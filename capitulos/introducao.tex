\xchapter{Introdu\c{c}\~{a}o}{}
\acresetall 

A gestação é um momento fisiológico na vida de uma mulher englobando diversas alterações hormonais, físicas, psicológicas e sociais, possibilitando na maioria das mulheres uma evolução sem intercorrências. Entretanto, é sabido que algumas gestantes são consideradas de alto risco por possuírem alguma condição clínica pré-existente, sofrerem algum agravo de tais condições ou até mesmo desenvolverem problemas que comprometam a evolução favorável da gravidez com condições médicas/obstétricas colocando em risco a saúde e o bem-estar tanto maternos quanto fetais \cite{melo2016,holness2018}. Em todo o mundo, cerca de 20 milhões de gestantes são consideradas de alto risco e mais de 800 morrem diariamente em decorrência das suas complicações perinatais \cite{bayrampour2013,lee2014,rodrigues2016}. 

Números alertam a gravidade e os perigos de gestações consideradas de alto risco, \citeonline{carvalheira2010} e a  \ac{WHO}\footnote{https://www.who.int/news-room/fact-sheets/detail/maternal-mortality} informam que o percentual das gestações de alto risco varia entre 6\% e 33\%, uma taxa preocupante para todos os países, sendo aceitável, de acordo com a \ac{OMS}, uma \ac{RMM} de 2 a 6 óbitos a cada 100 mil nascidos vivos. No Brasil, entre os anos de 1990 e 2015 a redução da \ac{RMM} foi de 143 para 62 óbitos por 100 mil nascidos vivos, uma diminuição de 56\%. Em 2015 foram registrados 1.738 casos de óbito materno por problemas relacionados à gestação, ao parto ou 42 dias após o parto\footnote{http://www.saude.gov.br/noticias/agencia-saude/43325-ministerio-da-saude-investe-na-reducao-da-mortalidade-materna}. Em 2016, houve uma redução de 16\%, com 1.463 óbitos registrados, sendo ainda, um número alto de mortalidade materna. Em Sergipe, a mortalidade materna teve uma redução de 13\% em 2016 quando comparada a 2011, sendo notificados 16 óbitos em gestantes ou 42 dias após o parto, enquanto em 2011 foram notificados 29 óbitos\footnote{https://www.saude.se.gov.br/?p=3229}. 

Para direcionar o diagnóstico e tratamento das doenças e/ou problemas que podem surgir durante a gestação, no Brasil, foi elaborado pelo \ac{MS} em conjunto com a Secretaria da Saúde e o Departamento de Ações Programáticas e Estratégicas o \ac{MTGAR}. Este manual visa padronizar as condutas  para que os atendimentos ocorram com uma atuação mais coerente, eficiente e de qualidade, descrevendo que a assistência pré-natal adequada, a qual visa acompanhar a gestante durante toda a gestação, e o preparo dos profissionais, podem identificar os riscos e impedir resultados perinatais desfavoráveis. De acordo com esse manual, algumas características são consideradas de risco para a gestação, como idade $<$15 ou $>$35 anos, altura $<$1,45m, \ac{IMC} $<$19 ou $>$30 kg/m², peso pré-gestacional $<$45 ou $>$75kg, situação conjugal insegura, baixa escolaridade, hábitos de vida, aborto, parto pré-termo, nuliparidade (nunca ter tido filhos) ou multiparidade (muitos partos anteriores), síndrome hemorrágica ou hipertensiva, diabetes gestacional, cirurgia uterina anterior (incluindo duas ou mais cesarianas), intervalo $<$2 ou $>$5 anos entre as gestações, hipertensão arterial, cardiopatias, pneumopatias, nefropatias, doenças autoimunes, etc. Além disso, o \ac{MTGAR} define a \ac{GAR} como aquela na qual a vida ou a saúde maternos e fetais têm maiores chances de serem atingidas de forma prejudicial do que a maioria das gestações e ressalta que uma gestação transcorrendo sem complicações pode passar a ser de risco a qualquer momento durante seu desenvolvimento ou mesmo no trabalho de parto.

%Para classificar a gestação como \ac{GAR}, no Brasil, foi elaborado pelo \ac{MS} em conjunto com a Secretaria da Saúde e o Departamento de Ações Programáticas e Estratégicas, o \ac{MTGAR} com o propósito de direcionar o diagnóstico e tratamento das doenças e/ou problemas que afetam a mulher durante a gravidez, bem como padronizar as condutas de atendimento, colaborando com uma atuação mais coerente, eficiente e de qualidade da equipe de serviço, descrevendo que a assistência pré-natal adequada a qual visa acompanhar a gestante durante toda a gestação e o preparo dos profissionais, podem identificar os riscos e impedir resultados desfavoráveis. De acordo com esse manual, algumas características são consideradas de risco, por exemplo: idade, altura, \ac{IMC}, peso pré-gestacional, histórico de aborto, diabetes gestacional, pré-eclampsia, eclâmpsia, etc. Além disso, o \ac{MTGAR} define a \ac{GAR} como aquela na qual a vida ou a saúde da mãe e/ou do feto e/ou do recém-nascido têm maiores chances de serem atingidas de forma prejudicial do que a maioria das gestações. 

O \ac{CDC}\footnote{https://www.cdc.gov/reproductivehealth/maternalinfanthealth/severematernalmorbidity.html} destaca que indicadores cardiovasculares, pulmonares, cirurgias, insuficiência renal aguda e sepse, são específicos para a morbidade materna. %alguns indicadores são específicos para a morbidade materna como: cardiovasculares, pulmonares, cirurgias, insuficiência renal aguda, sepse, entre outros%.
Além disso, sabe-se que o estado nutricional materno antes e durante a gestação interfere diretamente no prognóstico gestacional, uma vez que a desnutrição materna está associada a restrição de crescimento intrauterino e o excesso de peso está relacionado ao desenvolvimento de diabetes mellitus gestacional, pré-eclâmpsia, eclâmpsia, entre outros \cite{oliveira2018}. Devido a todos os fatores de risco e complicações existentes na \ac{GAR}, o \ac{MS} apresentou em um evento na \ac{OPAS} a meta de reduzir o número de mortalidade materna no Brasil para 30/100 mil nascidos vivos até 2030 \cite{valadares2018}. %Vale ressaltar ainda, que uma gestação transcorrendo sem complicações pode passar a ser de risco a qualquer momento durante seu desenvolvimento ou mesmo no trabalho de parto\footnote{http://bvsms.saude.gov.br/bvs/publicacoes/manual\_tecnico\_gestacao\_alto\_risco.pdf}.%

%Em face dessas informações, no Brasil, foi elaborado o \ac{MTGAR} pelo \ac{MS} em conjunto com a Secretaria da Saúde e o Departamento de Ações Programáticas e Estratégicas, com o propósito de direcionar o diagnóstico e tratamento das doenças e/ou problemas que afetam a mulher durante a gravidez, bem como padronizar as condutas de atendimento. 

Na literatura,  autores abordam a análise do perfil de GAR como maneira de auxiliar na redução do número de mortalidade materna. \citeonline{silva2014} afirmam que é necessário instaurar medidas que em nível primário, secundário e terciário possam interferir nas doenças de forma a evitar as complicações que possam comprometer a saúde materna e fetal. \citeonline{dalla2016} discutem que descrever o perfil epidemiológico das gestantes de alto risco em diferentes centros de referência, pode fornecer importantes informações no desenvolvimento de ações preventivas imediatas para as complicações da \ac{GAR}. \citeonline{anjos2014, melo2016} abordam a relevância  em conhecer o perfil das gestantes de alto risco, apresentam e estruturam as dificuldades que colaboram para aumentar os riscos gestacionais e o desenvolvimento de estratégias de saúde que possam minimizar os índices de \ac{GAR} e mortalidade perinatais, com uma assistência especializada para essas gestantes durante e após o período gestacional. 

%Portanto, pode-se perceber que ainda há espaço para realizar trabalhos abordando a descrição dos perfis de \ac{GAR}, podendo assim auxiliar no desenvolvimento de políticas e ações públicas que possam reduzir o número de gestações de alto risco, diminuindo assim, as suas consequências perinatais.  

Diante dos problemas acima mencionados, este trabalho visa desenvolver uma análise descritiva do perfil clínico e nutricional de gestantes de alto risco atendidas em ambulatório de nutrição materno infantil referência no município de Aracaju/SE. 
%A identificação desses perfis auxiliará na prevenção de fatores que levam à complicações gestacionais com a finalidade de verificar possíveis e eficazes medidas de prevenção, uma vez que esses fatores em sua maioria são passíveis de modificação. Logo, traçar o perfil de gestantes de alto risco pode ser um guia para informações que proporcionem medidas corretivas de tais complicações.%

\section{Motivação}\label{sec:intro:motivacao}
Embora pouco debatido, o tema mortalidade materna é complexo, pois é resultado de um conjunto de fatores biológicos, sociais, econômicos e culturais, demonstrando que não são óbitos acidentais, mas sim ocasionados por fatores que em sua maioria são evitáveis. Segundo o Guia de Vigilância Epidemiológica do Óbito Materno\footnote{http://bvsms.saude.gov.br/bvs/publicacoes/guia\_vigilancia\_epidem\_obito\_materno.pdf} a mortalidade materna é uma das mais graves violações dos direitos humanos das mulheres, principalmente porque em 92\% dos casos as causas podem ser evitadas e por acontecerem em países em desenvolvimento. 

O Brasil em 2013 apresentou uma redução nos índices de mortalidade materna, porém, em 2017 os números voltaram a crescer, passando de 62,1 para 64,5 óbitos maternos a cada 100 mil nascidos vivos, índice ainda elevado que contraria a meta firmada em evento da \ac{OPAS}. Em 2015, a \ac{ONU} divulgou que o Brasil era o quinto país mais lento na busca pela redução da taxa de mortalidade materna. Nos países em desenvolvimento, a taxa de mortalidade materna é extremamente alta quando comparada aos países desenvolvidos. Em 2015, foram declarados 239 óbitos por 100 mil nascidos vivos em países em desenvolvimento, enquanto nos países desenvolvidos foram declarados 12 óbitos por 100 mil nascidos vivos, ou seja, os países em desenvolvimento apresentam índices superiores a 1000\% se comparados com países desenvolvidos, demonstrando assim, que a disparidade no nível socioeconômico dos país interfere diretamente no desfecho perinatal\footnote{https://www.paho.org/bra/index.php?option=com\_content\&view=article\&id=5741:folha-informativa-mortalidade-materna\&Itemid=820}. %Dentre as principais complicações, 75\% das mortes maternas são em decorrência de hipertensão (pré-eclâmpsia ou eclâmpsia), hemorragias graves, infecções, complicações no parto e aborto de forma insegura.%

O estado nutricional pré-gestacional e gestacional, interfere diretamente no binômio mãe-filho. De acordo com \citeonline{oliveira2018}  a obesidade materna e o ganho de peso acima do recomendado estão associados ao aumento dos riscos de desfechos adversos para a mãe, tais como:  diabetes \textit{mellitus} gestacional, pré-eclâmpsia, parto prolongado, depressão e a indicação de cesariana, e para o concepto, com a variação do ganho de peso e o risco do nascimento \ac{GIG}, associado ao maior risco de sobrepeso, obesidade e alterações metabólicas durante a fase da infância e adolescência. Da mesma forma, o baixo peso materno ou o ganho de peso insuficiente durante a gestação ocasiona restrição de crescimento intrauterino, apresentando riscos ao neonato, tais como: nascimento de bebês \ac{PIG}, maiores índices infecções e mortalidade neonatal, prematuridade e baixos índices de Ápgar. \citeonline{paiva2012} abordam que além das complicações já mencionadas, a obesidade materna na gestação está associada a complicações no puerpério, principalmente as infecciosas

%De acordo com \citeonline{doddihal2015}, o fortalecimento dos serviços pré-natais auxiliam na percepção das condições de alto risco de forma precoce, favorecendo a prevenção de efeitos desfavoráveis a mãe e ao feto, além de proporcionar um serviço adequado e de qualidade. \citeonline{holness2018} apresenta uma revisão sistemática que aborda a importância do pré-natal adequado e da assistência especializada durante e após o parto na \ac{GAR}, além disso, retrata que a avaliação de risco gestacional é significante para determinar quando uma atenção maior deverá ser acrescida, minimizando assim a mortalidade materna e infantil. \citeonline{kiely2011} afirmam que além das condições médicas mais comuns como: obesidade, diabetes, pré-eclâmpsia e eclampsia, as condições sociodemográficas e comportamentais podem afetar a saúde perinatal, oferecendo risco à gestação. A melhor forma de obter resultados satisfatórios na prevenção da \ac{GAR} e da mortalidade materna, é através do incentivo e orientação de políticas públicas para o esclarecimento tanto dos profissionais quanto da população sobre a importância do pré-natal. Além disso, informações sobre quais gestantes estarão mais propensas a desenvolverem uma gestação de alto risco a partir da descrição do perfil epidemiológico associado a esses fatores, poderá auxiliar nas ações preventivas.

Diante de vários trabalhos disponíveis na literatura, percebe-se que há a espaço para desenvolver estudos  abordando o perfil epidemiológico das gestantes de alto risco a fim de auxiliar a maneira como ocorre o tratamento e a prevenção dos riscos gestacionais que podem ser modificados.
 
\section{Problema}\label{sec:intro:problema}
O problema que este trabalho investiga é o perfil clínico e nutricional de gestantes de alto risco atendidas em ambulatório de nutrição materno infantil referência no município de Aracaju/SE, visando proporcionar um auxílio na identificação de fatores de risco que possam auxiliar na redução das complicações da gestação de alto risco. Na maioria dos casos, a dificuldade em encontrar atendimento especializado durante a gestação caracteriza-se também como um fator de risco, uma vez que em sua maioria as causas da mortalidade materna são preveníveis durante o pré-natal.

Ressalta-se que há uma escassez de estudos epidemiológicos que abordem esta temática na região estudada, assim como a ausência de outros ambulatórios que promovam atendimento especializado para essa população, destacando uma oportunidade de pesquisa para o problema aqui abordado e com base nos resultados encontrados, há a possibilidade de desenvolver estratégias de intervenção eficazes, capazes de promover melhorias na saúde do binômio mãe-filho. 

\section{Objetivo}
O objetivo geral deste trabalho é identificar e traçar o  perfil clínico e nutricional das gestantes consideradas de alto risco atendidas em ambulatório referência de nutrição materno infantil no município de Aracaju/SE. 

Além do objetivo geral, este trabalho apresenta 3 objetivos específicos:

\begin{enumerate}[label=\textbf{{OE}{\arabic*}}:,itemindent=1em,leftmargin=3.5em]
    \item{Identificar os fatores de risco nutricional das gestantes atendidas no ambulatório;}
    \item{Compreender quais as condições clínicas mais prevalentes nas gestantes assistidas;}
    \item{Conhecer o perfil antropométrico pré-gestacional e atual das gestantes atendidas no ambulatório.}
\end{enumerate}

\section{Questões de Pesquisa}\label{sec:intro:questoes-pesquisa}


\begin{enumerate}[label=\textbf{{QP}{\arabic*}}:,itemindent=1em,leftmargin=3.5em]
\item{Qual o perfil clínico e nutricional de gestantes de alto risco atendidas em ambulatório referência de nutrição materno infantil?}

\end{enumerate}

\section{Metodologia}\label{sec:intro:metodologia}
A metodologia utilizada para o desenvolvimento deste trabalho foi uma análise descritiva do perfil das gestantes de alto risco atendidas no Ambulatório de Nutrição Materno Infantil localizado no Hospital Universitário da Universidade Federal de Sergipe, situado no município de Aracaju/SE. Esta metodologia será apresentada no Capítulo \ref{cap:procedimentos-metodologicos}. Contudo, foram realizadas algumas etapas para o desenvolvimento deste trabalho:

\begin{itemize}
    \item \textbf{Leitura e revisão da literatura:} Foi realizada uma revisão da literatura abordando trabalhos sobre a gestação de alto risco e o perfil epidemiológico das gestantes de alto risco;
    \item \textbf{Detecção de oportunidade de pesquisa:} Após a revisão de literatura realizada, foram encontrados trabalhos que abordam o perfil epidemiológico de gestantes de alto risco, apresentando análises descritivas desses perfis;
    \item \textbf {Desenho do estudo:} Estudo transversal descritivo;
    \item \textbf {Local do estudo:} Ambulatório referência de nutrição materno infantil, localizado no Hospital Universitário - UFS, no município de Aracaju/SE;
    \item \textbf {Participantes:} Participaram do estudo 17 gestantes de alto risco atendidas no ambulatório referência de nutrição materno infantil. Os critérios de inclusão foram: diagnóstico obstétrico de \ac{GAR} caracterizada por complicações médicas ou obstétricas durante a gravidez; encaminhamento médico para acompanhamento nutricional em \ac{GAR}; apresentar estado clínico e nutricional de risco; apresentar doenças pré-existentes que configurem risco para a gestação;
     \item \textbf {Realização da coleta de dados:} A coleta dos dados foi realizada no período de julho de 2017 a setembro de 2019. Para coletar os dados foi aplicado um questionário semiestruturado (ANEXO I) abordando as características socioeconômicas, histórico clínico e nutricional da gestante, antecedentes familiares, histórico obstétrico e da gestação atual, histórico dietético e dados antropométricos (IMC, peso e estatura). A aplicação desse questionário ocorreu na pré-consulta, onde a equipe do ambulatório de nutrição realizou a coleta das informações das gestantes encaminhadas ao atendimento. Todas as gestantes foram anteriormente diagnosticadas como de alto risco, atendidas no Ambulatório de Nutrição Materno Infantil do Hospital Universitário, em Aracaju/SE;
     \item \textbf {Análise dos dados:} Para a análise descritiva, os dados coletados foram tabulados utilizando o Excel como ferramenta auxiliar. As variáveis foram selecionadas de acordo com os estudos encontrados na leitura e revisão da literatura, sendo elas: nível de escolaridade e condição socioeconômica, idade, estado civil, peso pré-gestacional e gestacional, histórico de doenças pré-existentes, antecedentes familiares e histórico obstétrico;
     \item \textbf {Aspectos éticos:} O Hospital Universitário é um hospital escola e por isso quando um novo registro é efetuado, é assinado um termo de consentimento autorizando que as informações sejam utilizadas para pesquisa, de forma a preservar os dados pessoais dos pacientes. Este trabalho faz parte de um projeto de extensão da Universidade Federal de Sergipe e  respeita as normas éticas estabelecidas pelo Hospital Universitário, preservando as informações pessoais das gestantes atendidas.
\end{itemize}

\section{Contribuições Esperadas}\label{sec:intro:contribuicoes-esperadas}
A elaboração de um estudo descritivo que favoreça a compreensão da relevância da gravidez de risco, pode auxiliar na disseminação do conhecimento sobre esse tema e maneiras de prevenir de forma primária os riscos para o desenvolvimento das comorbidades associadas à gestação de alto risco. Com a construção desse trabalho, espera-se contribuir com estudos futuros nessa área, uma vez que ainda há espaço para mais pesquisas, principalmente no ambulatório de Nutrição Materno Infantil do Hospital Universitário de Aracaju no estado de Sergipe. Ainda assim, este trabalho possui algumas contribuições pontuais como:
\begin{itemize}
 \item{Revisão da literatura;}
 \item{Análise e identificação dos perfis de gestantes de alto risco;}
  \item{Compreensão das condições clínicas mais prevalentes das gestantes de alto risco assistidas;}
  \item{Estratégias de intervenção a partir das condições clínicas mais prevalentes;}
 \item{Banco de dados contendo os dados analisados.}
\end{itemize}

\section{Estrutura}

Neste capítulo foi apresentado uma visão geral do que será abordado neste trabalho. O restante desse trabalho será dividido da seguinte maneira: O Capítulo \ref{cap:referencial-teorico} aborda o referencial teórico no qual será discutido aspectos relevantes ao contexto deste trabalho. O Capítulo \ref{cap:procedimentos-metodologicos} apresenta os procedimentos metodológicos utilizados para realização deste estudo. O Capítulo \ref{cap:resultados} descreve os resultados encontrados, após a análise dos dados coletados. O Capítulo \ref{cap:discussao} discute aspectos dos resultados. O Capítulo \ref{cap:conlusao} conclui este trabalho apresentando as contribuições, limitações e trabalhos futuros.
